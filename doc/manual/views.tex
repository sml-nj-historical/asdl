%!TEX root = manual.tex
%
\chapter{Views}
\label{chap:views}

Views provide a general mechanism to customize the output of
\asdlgen{}.
Views allow description writers to annotate modules,
constructors, and types with directives or properties that are interpreted
by \asdlgen{}.
Currently \asdlgen{} properties that allow for the 
\begin{itemize}
  \item
    Inclusion of arbitrary user code in the resulting output.
  \item
    Automatic coercion of specific types into more efficient user
    defined representations.
  \item
    Addition of extra user defined attributes and initialization code .
  \item
    Specialized user defined routines to read and write data structures to the
    file system.
  \item
    Control over how the names of types, constructors, and modules names
    are mapped into the output language to resolve style issues and name space
    conflicts.
  \item
    Control over the tag values.
  \item
    Addition of documentation that describes the meaning of types
    constructors and modules.
\end{itemize}%

\section{Interpretation of View Entry Values}

See the section on \hyperref[sec:syntax]{Input Syntax} for details on
view the syntax and some basic view terminology. The view syntax associates
an arbitrary string whose interpretation depends on the property it is
assigned too. Currently there is a small set of standard interpretations.

\begin{description}
  \item[integer]
    An integral number in decimal notation.
  \item[string]
    A raw string.
  \item[boolean]
    A boolean value either ``\lstinline!true!'' or ``\lstinline!false!.''
  \item[qualified identifier]
    A possibly qualified identifier.
    For example ``\lstinline!M.t!'' or ``\lstinline!t!.''
    Qualified identifiers are language independent
    identifiers that are translated to the appropriate output language in a
    uniform way.
    For example \lstinline!M.t! would appear as \lstinline!M_t! in C and C++ but
    remain \lstinline!M.t! in SML, Java, and Haskell.
\end{description}%

\section{Adding User Code}
It is useful to be able to add arbitrary user code to the modules
produced by \asdlgen{}.
Modules have six properties that can be set to allow
the addition of user-code strings. 
\begin{description}
  \item[\texttt{interface\char`\_prologue}]\mbox{}\\
    Include text verbatim after the introduction of the base
    environment, but before any type defined in the module interface.

  \item[\texttt{interface\char`\_epilogue}]\mbox{}\\
    Include text verbatim after all types defined in the module
    interface have been defined.
	
  \item[\texttt{implementation\char`\_prologue}]\mbox{}\\
    Include text verbatim after the introduction of the base
    environment, but before any other implementation code is defined.
	
  \item[\texttt{implementation\char`\_epilogue}]\mbox{}\\
   Include text verbatim after all definitions defined in the module
   implementation.  

  \item[\texttt{suppress}]\mbox{}\\
   Default value is false. Do not produce any code for this module, 
   assume it's implementation is 
   written by hand. It's often a good idea to first generate code and then
   set the flag, so the generated code can be used as stubs for the 
   user implementation.

  \item[\texttt{is\char`\_library}]\mbox{}\\
   Default value is false. If true assume all types can be used
   as lists or options and generate any needed code, rather then 
   generating list and option code on demand. Useful for generating stubs.

\end{description}%

The precise meaning of interface and implementation for the different
languages is as follows
\begin{description}
  \item[C]
    The interface is the \texttt{.h} file and the implementation is the \texttt{.c}
    file.
  \item[C++]
    The interface is the \texttt{.hxx} file and the implementation is the \texttt{.cxx}
    file.
  \item[Java]
     Java has no notion of interfaces so this property is ignored. The
     \lstinline!implementation_prologue! is placed outside the class definition of a
     Java class that contains package wide global constants. The
     \lstinline!implementation_epilogue! is placed inside the class definition of a Java
     class that contains package wide global constants. 
  \item[ML]
    The interface is the generated signature the implementation is the structure.
  \item[Haskell]
    The interface is the export list of the module. The implementation is the
    module body.  
\end{description}%

\section{Other Properties}
\begin{description}
  \item[\texttt{doc\char`\_string}]\mbox{}\\
        All entities have this property. Its value is interpreted as a
        string. Currently only the \lstinline!--doc! command recognizes the
        property. It includes the property value in the HTML documentation
        produced for the module.
 
  \item[\texttt{source\char`\_name}]\mbox{}\\
        All entities have this property.  The value is interpreted as a
        string. Choose a different name for the type constructor or module
        in the output code. The name has no case restrictions. This is
        particularly useful when producing Java code on Windows NT/95 since
        the file system is case insensitive and types and constructors that
        differ only in case will cause problems.

  \item[\texttt{user\char`\_attribute}]\mbox{}\\
    Property of types only.	The value is interpreted as a qualified 
    identifier. Add a field called \lstinline!client_data! as an
    attribute to the type. The value is the qualified identifier that
    represents an arbitrary user type of the field. The \lstinline!client_data!
    field is ignored by the pickling code and does not appear in
    constructors. This property is currently only recognized when
    outputting C.

  \item[\texttt{user\char`\_init}]\mbox{}\\
    Property of types only. The value is interpreted as a qualified
    identifier. Call the function specified by the value before
    returning a the data structure created by a constructor function.
    This property is currently only recognized when outputting C.

  \item[\texttt{base\char`\_class}]\mbox{}\\
    Property of types only. The value is interpreted as a qualified
    identifier. The name of the class from which all classes generated
    for that type should inherit from. This property is recognized only
    when outputting C++ and Java.
    
  \item[\texttt{reader}]\mbox{}\\
   Property of types only. The value is interpreted as a qualified
   identifier.  Replace the body of the read pickle function for this
   type with a call to a function with the proper arguments. 

  \item[\texttt{writer}]\mbox{}\\
   Property of types only. The value is interpreted as a qualified
   identifier.  Replace the body of the writer pickle function for this
   type with a call to a function with the proper arguments. 

  \item[\texttt{enum\char`\_value}]\mbox{}\\
   Property of constructors only. The value is interpreted as an integer.
   Use this integer value as the \lstinline!internal! tag value for the
   constructor. The external pickle tag remains unchanged. 
   This property is recognized only when outputting C, C++, and Java. 

\end{description}%

\section{Choosing a Different Representation}

\begin{quote}\begin{lstlisting}[language=ASDL]
module IntMap {
  int_map = (int size,entries map)
  entries = (entry* entries)
  entry   = (int key, int value)
}
\end{lstlisting}\end{quote}%
The above is one possible abstract description of a mapping from integers to
integers.  It would be more efficient to <em/implement/ such a mapping as
a binary tree. Described as with the ASDL definition below.
\begin{quote}\begin{lstlisting}[language=ASDL]
module IntMap {
  int_map = (size int,map tree)
     tree = Node(int key,int value,tree left,tree right)
          | Empty
}
\end{lstlisting}\end{quote}%

Although this is a much more efficient representation it exposes
implementation details. If we decided to change the implementation of
\lstinline!int_maps! to use a hash table the all other clients that use our type
will have to be updated.

The view properties \lstinline!natural_type!, \lstinline!natural_type_con!, \lstinline!wrapper!,
and \lstinline!unwrapper! provide a general mechanism to choose a different
more efficient representation through coercion functions. All of these
properties apply to types only and are interpreted as qualified identifiers.

\begin{description} 
  \item[\texttt{natural\char`\_type}] \mbox{} \\
    The type to use in place of the original type in all the resulting code. 
    Supported by all output languages.

  \item[\texttt{natural\char`\_type\char`\_con}] \mbox{} \\
    A unary type constructor to apply to the old type to get a new type to
    use in all the resulting code. e.g. ref in ML to make a type
    mutable. Supported by ML and Haskell.  <em/Support for C++ templates
    will be added in the near future./

  \item[\texttt{wrapper}] \mbox{} \\
    A function to convert the new type to the old type when writing the
    pickle. Supported by all output languages.

  \item[\texttt{unwrapper}] \mbox{} \\
    A function to convert the old type to the new type when reading the pickle. 
    Supported by all output languages. 
\end{description}%

When using \lstinline!natural_type! and \lstinline!natural_type_con! the automatically
generated type definitions for the original type still remain, but all other
references to the original type in constructors, picklers, and other type
definitions that referred to it are replaced with the new type. The original
definition must remain to support pickling of the type. Pickling is achieved
by appropriately coercing the new type to the old type and vice versa with
functions specified by \lstinline!wrapper! and \lstinline!unwrapper! properties.

\subsection{Examples}
%% Need more examples/

\begin{quote}\begin{lstlisting}[language=ASDL]
module Slp {

 real = (int mantissa, int exp)
...
  exp = Id(identifier id)
      | Num(int v)
      | Op(exp lexp,binop?,exp rexp)
       attributes (real? value)	
...

}

view C {

-- represent reals as a double

  Slp.exp <= {
   natural_type : my_real
   wrapper      : real2my_real
   unwrapper    : my_real2real
  }
  
  module Slp <= {
  interface_epilogue : typedef double my_real_ty
  
  implementation_prologue
%%
  my_real_ty real2my_real(Slp_real_ty x) {
    /* hairy code to actually do this */
  }

  Slp_real_ty my_real2real(my_real_ty x) {
    /* hairy code to actually do this */
  }
%% 
  } 
}

view SML {
   -- unpickle exp trees as a mutable type
    Slp.exp <= {
       natural_type_con : ref
       wrapper          : !
       unwrapper        : ref
    }

}
\end{lstlisting}\end{quote}%
