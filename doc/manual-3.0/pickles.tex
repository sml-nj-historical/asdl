%!TEX root = manual.tex
%
\chapter{Pickles}
\label{chap:pickles}

One of the most important features of \asdlgen{} is that it automatically
produces functions that can read and write the data structures it generates
to and from a platform and language independent external
representation.
This process of converting data structures in memory into a
sequence of bytes on the disk is referred to as \emph{pickling}.
Since it is possible to generate data structures and pickling code for
any of the supported languages from a single \asdl{} specification,
\asdlgen{} provides an easy
and efficient way to share complex data structures among these languages.

The \asdl{} pickle format requires that both the reader and writer
of the pickler agree on the type of the pickle.
Other than constructor tags for sum types, there is no explicit type
information in the pickle.
In the case of an error the behavior is undefined.
It is also important that the pickling/unpickling to/from files, that the
files be opened in binary mode to prevent line feed translations from corrupting
the pickle.

\section{Binary Pickle Format}

Since \asdl{} data structures have a tree-like form, they can be represented
linearly with a simple prefix encoding.

\begin{description}
  \item[int/uint] \mbox{}\\
    The \lstinline!int! and \lstinline!uint! types provide 30-bits of signed/unsigned
    precision encoded in one to four bytes.
    The top two bits of the first byte specifies the number of additional bytes in
    the encoding.
    Thus values in the range -32 to 31 (0 to 63 unsigned) can be encoded in one byte,
    -2048 to 2047 (0 to 2095 unsigned) in two bytes, \etc{}

  \item[bool] \mbox{}\\
    Boolean values are represented by \lstinline!0! (false) or \lstinline!1! (true)
    and are encoded as \lstinline!uint! values.

  \item[integer] \mbox{}\\
    Since \asdl{} integers are intended to be of infinite precision they are
    represented with a variable-length, signed-magnitude encoding.
    The eight bit of each byte indicates if a given byte is the last byte of
    the integer encoding.
    The seventh bit of the most significant byte is used to determine the
    sign of the value.
    Thus, numbers in the range of -63 to 63 can be encoded in one byte.
    Numbers outside of this range require an extra byte for every seven bits
    of precision required.
    Bytes are written out in little-endian order.
    If most integer values tend to be values near zero, this encoding of integers
    will use less space than a fixed precision representation.

  \item[string] \mbox{}\\
    Strings are represented with a length-header that describe how many more
    8-bit bytes follow for the string and then the data for the string in bytes.
    The length-header is encoded as a \lstinline!uint! value, thus strings are limited
    to 1,073,741,823 characters.

  \item[identifier] \mbox{}\\
    Identifiers are represented as if they were strings. 

  \item[product types] \mbox{}\\ 
    Product types are represented sequentially without any tag.
    The fields of the product type are packed from left to right.

  \item[sum types] \mbox{}\\
    Sum types begin with a unique tag to identify the constructor
    followed sequentially by the fields of the constructor.
    Tag values are assigned based on the order of constructor definition in the
    description.
    The first constructor has a tag value of zero.
    Fields are packed left to right based of the order in the definition.
    If there are any attribute values associated with the type, they are packed left to right
    after the tag but before other constructor fields.
    The tag is encoded as a \lstinline!uint! value.

  \item[sequence types] \mbox{}\\
    Sequence types are represented with an integer length-header followed by
    that many values of that type.
    The length-header is encoded as a \lstinline!uint! value.

  \item[option types] \mbox{}\\
    Optional values are preceded by an integer header that is either one or zero.
    A zero indicates that the value is empty and no more data follows.
    A one indicates that the next value is the value of the optional value.
    The header is encoded as a \lstinline!uint! value.

  \item[primitive types] \mbox{}\\
    User-defined primitive types are pickled/unpickled by user-provided functions (see
    \secref{sec:primitive-syntax}).
\end{description}%

\section{S-expression Format}
It is also possible to generate a text-based representation of pickles in S-Expression
syntax.


\section{XML Pickle Format}

