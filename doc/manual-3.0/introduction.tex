%!TEX root = manual.tex
%
\chapter{Introduction}
\label{chap:introduction}

\thispagestyle{empty}

The \emph{Abstract Syntax Description Lanuguage} (\asdl{}) is a language
designed to describe the tree-like data structures used in compilers.
Its original purpose was to provide a method for
compiler components written in different languages to
interoperate~\cite{usenix:zephyr-asdl}, but it has also been
used to support communicating information between separate runs
of a compiler.
\asdl{} makes it fairly easy for applications written in a
variety of programming languages to communicate complex recursive data
structures. 

\asdlgen{} is a tool that takes \asdl{} descriptions and produces
implementations of those descriptions in a variety of languages.
\asdl{} and \asdlgen{} together provide the following advantages
\begin{itemize}
  \item Concise descriptions of important data structures.
  \item Automatic generation of data structure implementations for
    \asdlgen{}-supported languages.
  \item Automatic generation of functions to read and write the data
    structures to disk in a machine and language independent way.
\end{itemize}%

\asdl{} descriptions describe the tree-like data structures such as
abstract syntax trees (ASTs) and compiler intermediate representations
(IRs).
Tools such as \asdlgen{} automatically produce the equivalent
data structure definitions for the supported languages.
\asdlgen{} also produces functions for each language that read and
write the data structures to and from a platform and language
independent sequence of bytes.
The sequence of bytes is called a \emph{pickle}.

\asdl{} was originally developed in the 1990's by Daniel Wang as part of the
\emph{National Compiler Infrastructure} project at Princeton University.
That \asdl{} implementation has not kept up with the significant changes
in many of its target languages (\eg{}, \Cplusplus{}, \haskell{}, \java{},
\etc{}), so it was time for a rewrite.\footnote{
  Version~2.0 of \asdl{} is still available from
  \url{https://sourceforge.net/projects/asdl}
  and the~2.0 version of the manual (converted to \LaTeX{}) is included
  in the documentation of this system.
}
Version~3.0 of \asdl{} and \asdlgen{} is a complete reimplementation of the
system, with the primary purpose of supporting the \smlnj{} compiler.
As such, it currently only supports generating picklers in \sml{} and
modern \Cplusplus{}, although other languages may be added as time
permits.

\section{Changes From Version 2.0}
\label{sec:changes}

The following is a list of the major changes from the 2.0 version of \asdl{}
and \asdlgen{}:
\begin{itemize}
  \item
    The primitive types were extended and changed.
    The \lstinline!bool! type was added, the type name of arbitrary precision integers
    was changed to \lstinline!integer!, and the types \lstinline!int! and \lstinline!uint!
    were added to represent small integers.
  \item
    Various changed were made to the binary encoding of pickles.
  \item
    Currently only two target languages are supported: \sml{} and \Cplusplus{}.
  \item
    The generated \Cplusplus{} code targets the 2011 standard and uses the
    \Cplusplus{} STL (\eg{}, \lstinline[language=C++]!std::vector<>! for \asdl{} sequences).
  \item
    Data can be pickled/unpickled to/from memory, as well as files.
    This change affects the requirements for implementing primitive modules.
  \item
    Alias-type definitions were added to the ASDL syntax.
  \item
    Include directives were added to support splitting specifications into multiple
    files (and the sharing of common specifications).
\end{itemize}%
