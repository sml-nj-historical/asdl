%!TEX root = manual.tex
%
\chapter{Input Syntax}
\label{chap:syntax}

This section describes the syntax of the input language to \asdlgen{}.
The syntax is described using EBNF notation.
Literal terminals are typeset in bold and enclosed in single quotes.
Optional terms are enclosed in square brackets and terms that are
repeated zero or more times are enclosed in braces.
The description is broken up in smaller subsection.
Each subsection describes a fragment of
the syntax and its meaning.

\section{Lexical Tokens}

\begin{figure}[t]
  \begin{quote}
    \begin{grammar}
      <upper>     ::= `A' | ... | `Z'

      <lower>     ::= `a' | ... | `z'

      <alpha>     ::= `_' | <upper> | <lower>

      <alpha-num> ::= <alpha> | `0' | ... | `9'

      <typ-id>    ::= <lower> \{ <alpha-num> \}

      <con-id>    ::= <upper> \{ <alpha-num> \}

      <id>        ::= <typ-id> | <con-id>
    \end{grammar}
  \end{quote}
  \caption{Lexical rules for \asdl{} terminals}
  \label{fig:lexical-syntax}
\end{figure}%

Notice that identifiers for types, \synt{typ-i}, are in a different
lexical class from identifiers for constructors
\synt{con-id}. Constructor identifiers must start with one uppercase
letter, while type identifiers must begin with one lowercase
letter. All identifiers must begin with a letter.  Comments (not shown
in the above syntax) begin with ``\lstinline[language=ASDL]!--!''
and continue to the end of the line.

\section{Module Syntax}

\begin{figure}[t]
  \begin{quote}
    \begin{grammar}
      <module>  ::=  `module' <id> [ <imports> ] `{' <definitions> `}'

      <imports> ::=  `(' \{ `imports' <id> \} `)'
    \end{grammar}
  \end{quote}
  \caption{\asdl{} module syntax}
  \label{fig:module-syntax}
\end{figure}%

An ASDL module declaration consists of the keyword \lit{module}
followed by an identifier, an optional set of imported modules, and a
sequence of type definitions enclosed in braces. For example the
following example declares modules \lstinline[language=ASDL]!A!,
\lstinline[language=ASDL]!B!, and \lstinline[language=ASDL]!C!.
\lstinline[language=ASDL]!B! imports types from \lstinline[language=ASDL]!A!.
\lstinline[language=ASDL]!C! imports types from both \lstinline[language=ASDL]!A! and
\lstinline[language=ASDL]!B!.
Imports cannot be recursive; for example, it is an error for \lstinline[language=ASDL]!B! to
import \lstinline[language=ASDL]!C!, since \lstinline[language=ASDL]!C!
imports \lstinline[language=ASDL]!B!.
\begin{quote}\begin{lstlisting}[language=ASDL]
 module A { ... } 
 module B (imports A) { ... }
 module C (imports A 
           imports B) { ... }
\end{lstlisting}\end{quote}% 

To refer to a type imported from another module the type must
\emph{always} be qualified by the module name from which it is
imported.
The following declares two different types called ``\texttt{t}.''
One in module \texttt{A} and one in module \texttt{B}.
The type ``\texttt{t}'' in module \texttt{B} defines a type ``\texttt{t}'' that
recursively mentions itself and also references the type ``\texttt{t}'' imported
from module \texttt{A}.
\begin{quote}\begin{lstlisting}[language=ASDL]
module A { t = ... } 
module B (imports A) { t = T(A.t, t) | N  ... }
\end{lstlisting}\end{quote}% 

\section{Type Definitions}

\begin{figure}[t]
  \begin{quote}
    \begin{grammar}
      <definitions>  ::=  \{ <typ-id> `=' <type> \}

      <type>         ::= <sum-type> | <product-type>

      <product-type> ::= <fields>

      <sum-type>     ::= <constructor> \{ `|' <constructor> \} [ `attributes' <fields> ]

      <constructor>  ::= <con-id> [ <fields> ]

      <fields>       ::= `(' \{ <field>  `,' \} <field> `)'

      <field>        ::= [ id `.' ] <typ-id> [`?' | `*' ]  [ <id> ]
    \end{grammar}
  \end{quote}
  \caption{\asdl{} type definition syntax}
  \label{fig:type-syntax}
\end{figure}%

All type definitions occur within a module.  They begin with a type
identifier, which is the name of the type. The name must be unique within the
module. The order of definitions is unimportant. When translating type
definitions from a module they are placed in what would be considered a
module, package, or name-space of the same name. If a output language does
not support such features and only has one global name space the module name
is used to prefix all the globally exported identifiers.

Type definitions are either product types, which are simple record definitions,
or sum type, which represent a discriminated union of possible values. Unlike
sum types, product types cannot form recursive type definitions, but they can
contain recursively declared sum types.

\subsection{Primitive Types}
There are only three primitive types in ASDL. They
are \lstinline[language=ASDL]!string!, \lstinline[language=ASDL]!int!, and \lstinline[language=ASDL]!identifier!. The \lstinline[language=ASDL]!string! type
represents length encoded strings of bytes. The \lstinline[language=ASDL]!int! type represents
arbitrary precision signed integers. The \lstinline[language=ASDL]!identifier! type represents
atomic printable names, which support fast equality testing, analogous to
LISP symbols.

\subsection{Product Types}
Product types are record or tuple declarations. They consist of a non-empty 
sequence of fields separated by commas enclosed in parenthesis. For
example 
\begin{quote}\begin{lstlisting}[language=ASDL] 
pair_of_ints = (int, int) 
\end{lstlisting}\end{quote}% 
declares a new type \lstinline[language=ASDL]!pair_of_ints! that consists of two integers.

\subsection{Sum Types}

Sum types are the most useful types in ASDL. They provide concise notation
used to describe a type that is the tagged union of a finite set of other
types.  Sum types consists of a series of constructors separated by a
vertical bar. Each constructor consist of a constructor identifier followed
by an optional sequence of fields similar to a product type. 

Constructor names must be unique within the module in which they are
declared. Constructors can be viewed as functions who take some number of
arguments of arbitrary type and create a value belonging to the sum type in
which they are declared. For example
\begin{quote}\begin{lstlisting}[language=ASDL]
module M {
  sexpr = Int(int)
	| String(string)
	| Symbol(identifier)
	| Cons(sexpr, sexpr)
	| Nil
}
\end{lstlisting}\end{quote}%
declares that values of type \lstinline[language=ASDL]!sexpr! can either be constructed from an
\lstinline[language=ASDL]!int! using the \lstinline[language=ASDL]!Int! constructor or a \lstinline[language=ASDL]!string! from a \lstinline[language=ASDL]!String!
constructor, an \lstinline[language=ASDL]!identifier! using the \lstinline[language=ASDL]!Symbol! constructor, or from
two other \lstinline[language=ASDL]!sexpr! using the \lstinline[language=ASDL]!Cons! constructor, or from no arguments
using the \lstinline[language=ASDL]!Nil! constructor. Notice that the \lstinline[language=ASDL]!Cons! constructor
recursively refers to the \lstinline[language=ASDL]!sexpr! type. ASDL allows sum types to be
mutually recursive. Recursion however is limited to sum types defined within
the same module.

\subsubsection{The Rosetta Stone for Sum Types}
\label{sec:rosetta-stone}

In languages that have algebraic data types sum types are equivalent to the
\lstinline[language=ASDL]!datatype! and \lstinline[language=ASDL]!data! declarations in ML and Haskell. In Algol like
languages they are equivalent to tagged \lstinline[language=ASDL]!unions! in C or variant records
in Pascal. In class based object oriented languages sum types are equivalent
to an abstract base class that represents the type of a family of subclasses
one for each constructor of the type. The previous example written in ML would
be
\begin{quote}\begin{lstlisting}[language=SML]
structure M =
  struct
    datatype sexpr
      = Int of (int)
      | String of (string)
      | Symbol of (identifier)
      | Cons of (sexpr * sexpr)
      | Nil
  end
\end{lstlisting}\end{quote}%
and in \Cplusplus{} it translates to
\begin{quote}\begin{lstlisting}[language=c++]
namespace M {

    struct sexpr {
        enum tag {
            _Int, _String, _Symbol, _Cons, _Nil
        };
        tag _tag;
        sexpr (tag t) : _tag(t) { }
        virtual ~sexpr ();
    };

    struct Int : public sexpr {
        int _v1;
        Int (int v) : sexpr(sexpr::_Int), _v1(v) { }
        ~Int () { }
    };

    struct String : public sexpr {
        std::string _v1;
        String (const char *v) : sexpr(sexpr::_String), _v1(v) { }
        String (std::string const &v) : sexpr(sexpr::_String), _v1(v) { }
        ~String () { }
    };

    struct Symbol : public sexpr { ... };

    struct Cons : public sexpr { ... };

    struct Nil : public sexpr { ... };

}
\end{lstlisting}\end{quote}%

\subsubsection{Sum Types as Enumerations}
\label{sec:enumerations}

Sum types, which consist completely of constructors that take no arguments,
are often treated specially and translated into static constants of a
enumerated value in languages that support them.
For example, the following \asdl{} specification:
%
\begin{quote}\begin{lstlisting}[language=ASDL]
module Op {
  op = PLUS | MINUS | TIMES | DIVIDE 
}
\end{lstlisting}\end{quote}%
%
Is translated into \Cplusplus{} as follows:
%
\begin{quote}\begin{lstlisting}[language=c++]
namespace M {
    enum op {
        PLUS, MINUS, TIMES, DIVIDE
    };
}
\end{lstlisting}\end{quote}%

\subsection{Field Labels}
Field declarations can be as simple as a single type identifier, or they can
be followed with an optional label.
If any field in a product type (or constructor type argument) has a label, then
all of the fields must have labels.
Labels aid in the readability of
descriptions and are used by \asdlgen{} to name the fields of records
and classes for languages.
For example the declarations
\begin{quote}\begin{lstlisting}[language=ASDL]
module Tree {
  tree = Node(int, tree, tree)
       | EmptyTree
}
\end{lstlisting}\end{quote}%
can also be written as
\begin{quote}\begin{lstlisting}[language=ASDL]
module Tree {
  tree = Node(int value, tree left, tree right)
       | EmptyTree
}
\end{lstlisting}\end{quote}%

When translating the first definition without labels 
into \Cplusplus{} one would normally get
\begin{quote}\begin{lstlisting}[language=c++]
namespace Tree {

    class tree {
      public:
        enum tag_t {
            _Node, _EmptyTree
        };
        tag_t tag () const { return this->_tag; }
        bool is_Node () const { return this->_tag = _Node; }
        bool is_EmptyTree () const { return this->_tag = _EmptyTree; }
        virtual ~tree ();
      protected:
        tag _tag;
        tree (tag t) : _tag(t) { }
    };

    struct Node : public tree {
        int _v1;
        tree *_v2;
        tree *_v3;
        Node (int v1, tree *v2, tree * v3)
          : tree(tree::_Node), _v1(v1), _v2(v2), _v3(v3) { }
        ~Node () { }
    };
    
    struct EmptyTree : public tree {
        EmptyTree () : tree(tree::_EmptyTree) { }
        ~EmptyTree () { }
    };
}
\end{lstlisting}\end{quote}%
with labels one would get
\begin{quote}\begin{lstlisting}[language=c++]
namespace Tree {

    ...
    
    struct Node : public tree {
        int value;
        tree *left;
        tree *right;
        Node (int v1, tree *v2, tree * v3)
          : tree(tree::_Node), value(v1), left(v2), right(v3) { }
        ~Node () { }
    };
    
    ...
}
\end{lstlisting}\end{quote}%

For the \sml{} target, product types without labels are translated to tuples, while those with
labels are translated to 
\subsection{Type Qualifiers}

The type identifier of a field declaration can also be qualified with either
a sequence (\lit{*}) or option (\lit{?}) qualifier.
The sequence qualifier
is an abbreviation that stands for a sequence or list of that given type,
while the option qualifier stands for a type whose value may be
uninitialized.
Sequence types are equivalent to the lists or arrays of a
fixed type.
Option types are equivalent to the \lstinline[language=ASDL]!option!
type in \sml{}.
%\asdlgen{} provides various different translation schemes for handling option
%and sequence types in languages that do not have parametric polymorphism. 
%See \chapref{chap:invocation} for details.

\subsection{Attributes}
Sum types may optionally be followed by a set of
attributes.
Attributes provide notation for fields common to all the
constructors in the sum type. 
\begin{quote}\begin{lstlisting}[language=ASDL]
module M {
  pos = (string file, int linenum, int charpos)
  sexpr = Int(int)
	| String(string)
	| Symbol(identifier)
	| Cons(sexpr, sexpr)
	| Nil
        attribute(pos p)
}
\end{lstlisting}\end{quote}%
adds a field of type \lstinline[language=ASDL]!pos! with label \lstinline[language=ASDL]!p! to all the constructors in \lstinline[language=ASDL]!sexpr!.
Attribute fields are treated specially when
translated.
For example in \Cplusplus{} code the attribute field is defined in the base class for the sum type.

\section{Primitive Modules}
\label{sec:primitive-module}

\begin{figure}[t]
  \begin{quote}
    \begin{grammar}
      <primitive-module> ::= `primitive' `module' <id> `{' \{ <id> \} `}'
    \end{grammar}%
  \end{quote}%
  \caption{\asdl{} primitive module syntax}
  \label{fig:prim-module-syntax}
\end{figure}%

Primitive modules (see \figref{fig:prim-module-syntax}) provide a way to introduce abstract
types that are defined outside
of \asdl{} and which have their own pickling and unpickling code.
For example, we might want to include GUIDs (16-byte globally-unique IDs) in our pickles.
We can do so by first defining a primitive module \lstinline!Prim!:
%
\begin{quote}\begin{lstlisting}[language=ASDL]
primitive module Prim { guid }
\end{lstlisting}\end{quote}%
%
Then, depending on the target language, we define
supporting code to read and write guids from the byte stream.
In \sml{}, we would define three modules:
\begin{enumerate}
  \item \lstinline[language=SML]!structure PrimPickle!, which defines the representation of the
    \lstinline[language=SML]!guid! type.
  \item
    \lstinline[language=SML]!structure PrimPickleUtil!, which defines a function for calculating
    the number size of a pickle (in bytes) and functions for converting between GUIDs and
    vectors of bytes.
  \item
    \lstinline[language=SML]!structure PrimPickleIO!, which defines functions for reading
    and writing GUIDs on binary I/O streams.
\end{enumerate}%
The \sml{} implementation of these modules could be written as follows:
\begin{quote}\begin{lstlisting}[language=SML]
structure PrimPickle : sig
    type guid
  end = struct
    type guid = GUID.guid
  end

structure PrimPickleUtil : sig
    val guidSize : PrimPickle.guid -> size
    val guidToBytes : PrimPickle.guid -> Word8Vector.vector
    val guidFromBytes : { input : int -> Word8Vector.vector option }
          -> PrimPickle.guid
  end = struct
    fun guidSize _ = 16
    fun guidToBytes = GUID.toBytes
    fun guidFromBytes { input } = (case input 16
           of NONE => raise Fail "bogus GUID"
	        | (SOME v) => (case GUID.fromBytes v
                 of NONE => raise Fail "bogus GUID"
	              | SOME g => g
                (* end case *))
	      (* end case *))
  end

structure PrimPickleIO : sig
    val guidInput : BinIO.instream -> guid
    val guidOutput : BinIO.outstream * guid -> unit
  end = struct
    fun guidInput strm = (case GUID.fromBytes(BinIO.inputN(strm, 16))
	       of NONE => raise Fail "bogus GUID"
	        | (SOME g) => g
	      (* end case *))
    fun guidOutput (strm, g) = BinIO.output(strm, GUID.toBytes g)
  end
\end{lstlisting}\end{quote}%
(assuming that the \lstinline!GUID! module implements the application's representation
of GUIDs).

For \Cplusplus{}, a primitive module requires a corresponding header file that declares
the primitive types and instances of the overloaded \lstinline[language=c++]!<<! and
\lstinline[language=c++]!>>! operators on the primitive types.  These declarations should
all be in a \lstinline[language=c++]!namespace! with the type name of the primitive module.
For example, the module from above would require the provision of a \texttt{Prim.hxx} header
file that contained something like the following code:
%
\begin{quote}\begin{lstlisting}[language=c++]
#include <iostream>
#include "guid.hxx"

namespace Prim {

    typedef GUID::guid guid;

    std::istream &operator>> (istream &is, guid &g);
    std::ostream &operator<< (ostream &os, guid const &g);

}
\end{lstlisting}\end{quote}%
(assuming that the \texttt{guid.hxx} header defines the application's representation
of GUIDs).
%For the \Cplusplus{} target, one should use string streams (in binary mode)
%to support pickling/unpickling of to/from memory.


\section{View Syntax}

\begin{figure}[t]
  \begin{quote}
    \begin{grammar}
      <view>        ::= `view' <id> `{' \{ <view-entry> \} `}'

      <view-entry>  ::= <view-entity>  `<=' <id> <lit-txt>
         \alt{} `{' {<view-entity>} `}' `<=' <id> <lit-txt>
         \alt{} `<=' <id> `{' \{ <view-entity> <lit-txt> \} `}'
         \alt{} <view-entity> `<=' `{' \{ <id> <lit-txt> \} `}'

      <view-entity> ::= <id> `.' <typ-id>
         \alt{} <id> `.' <con-id>
         \alt{} `module' <id>

      <lit-txt>     ::= <see discussion below>
    \end{grammar}
  \end{quote}
  \caption{\asdl{} view syntax}
  \label{fig:view-syntax}
\end{figure}%

Views are an extensible system of annotating modules, types, and
constructors to aid in the conversion of ASDL declarations into a
particular output language.
The syntax of view declarations is given in \figref{fig:view-syntax}.
This section covers the syntax of views, but leaves the semantics to
\chapref{chap:views}.

\subsection{Basic Syntax}
Views are named and consist of series of entries.
Each entry conceptually
begins with a fully qualified type, constructor, or module name. Following
that name is a key value pair.
The meaning of the entry is to add a
particular key value pair into an internal table associated with the type,
constructor, or module.
The type, constructor or module is referred to as
the view entity the key is referred to as the view property.
The value is an arbitrary string.
In the grammar above the arbitrary string is represented
by the \synt{lit-txt} non-terminal, where \synt{lit-txt} is either a ``\lstinline!:!''
followed by some arbitrary text that continues to the end of the line or a
``\lstinline!%%!'' terminated by a ``\lstinline!%%!'' at the beginning of a line by itself. 
Text include using the ``\lstinline!:!'' notation will have trailing and leading 
whitespace removed. For example
%
\begin{quote}\begin{lstlisting}[language=ASDL]
view Doc {
  module  M <= doc_string
%%
  Types for representing LISP s-expressions.
%%
  M.sexpr  <= doc_string : s-expressions 
  M.Int    <= doc_string : s-expression constructor
  M.Symbol <= doc_string : s-expression constructor
  M.Cons   <= doc_string : s-expression constructor
  M.Nil    <= doc_string : s-expression constructor
}

view Java {
 M.sexpr <= source_name : Sexpr
 M.sexpr <= base_class  : MyClass
}
\end{lstlisting}\end{quote}%
%
associates with the module \lstinline[language=ASDL]!M! the type
\lstinline[language=ASDL]!M.sexpr! and the
constructor \lstinline[language=ASDL]!M.Int!
strings that will be added to the automatically generated documentation
produced by the \texttt{--doc} command of \asdlgen{}.
(\emph{In future we will probably dump them in comments in the output code too.})
The view named \lstinline[language=ASDL]!Java! causes the type
\lstinline[language=ASDL]!M.sexpr! to be renamed \lstinline[language=ASDL]!Sexpr! when
generating Java output, and causes the abstract class normally generated to
inherit from \lstinline[language=ASDL]!MyClass!. 

There can be many views with the same name. The entries of two views
with the same name are merged and consist of the union of the entries
in both. It is an error, for two views of the same name to assign
different values to the same property of an entity. \emph{Currently
this error isn't checked for so you randomly get one or the other.}

\subsection{Sugared Syntax}
The above example shows only the simplest syntax to define view entries.
There are also three sugared versions that remove some of the redundancy of the simple
syntax. The first sugared version allows the assignment of the same
property value pair to a set of entities. The next sugared version allows
assigning to different entities different values for a fixed property.
The final sugared version allows the assignment of different property
value pairs to the same entity. Examples of the sugared notation are shown
below in their respective order.
\begin{quote}\begin{lstlisting}[language=ASDL]
view Doc {
 { M.Int  M.Symbol  
   M.Cons M.Nil } <= doc_string : s-expression constructor

 <= doc_string {
  module  M 
%%
  Types for representing LISP s-expressions.
%%
  M.sexpr : s-expressions 
  }
}

view Cxx {
 M.sexpr <= {
   source_name : Sexpr
   base_class  : MyClass
 }
}
\end{lstlisting}\end{quote}%
