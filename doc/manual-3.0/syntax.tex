%!TEX root = manual.tex
%
\chapter{ASDL Syntax}
\label{chap:syntax}

This section describes the syntax of the input language to \asdlgen{}.
The syntax is described using EBNF notation.
Literal terminals are typeset in bold and enclosed in single quotes.
Optional terms are enclosed in square brackets and terms that are
repeated zero or more times are enclosed in braces.
Each section describes a fragment of the syntax and its meaning.

\section{Lexical Tokens}

The lexical conventions for \asdl{} are given in \figref{fig:lexical-syntax}.
ASDL is a case-sensitive language and, furthermore, classifies identifiers
into initial-lower-case (\synt{lc-id}) and initial-upper-case (\synt{uc-id}) identifiers.
Type identifiers are initial-lower-case, while constructor identifiers are initial-upper-case.
Module and field identifiers can be either upper or lower-case.

\begin{figure}[t]
  \begin{quote}
    \begin{grammar}
      <upper>     ::= `A' | ... | `Z'

      <lower>     ::= `a' | ... | `z'

      <alpha>     ::= `_' | <upper> | <lower>

      <alpha-num> ::= <alpha> | `0' | ... | `9'

      <lc-id>     ::= <lower> \{ <alpha-num> \}
      
      <uc-id>     ::= <upper> \{ <alpha-num> \}

      <id>        ::= <lc-id> | <uc-id>
                  
      <typ-id>    ::= <lc-id>

      <con-id>    ::= <uc-id>

      <comment>   ::= `--' \{ <text-character> \} <end-of-line>

      <text>      ::= `:' \{ <text-character> \} <end-of-line>
               \alt{} `\%\%' \{ <text-character> | <end-of-line> \} <end-of-line> `\%\%'
    \end{grammar}
  \end{quote}
  \caption{Lexical rules for \asdl{} terminals}
  \label{fig:lexical-syntax}
\end{figure}%

Comments begin with ``\lstinline[language=ASDL]!--!''
and continue to the end of the line.

Verbatim text is denoted by \synt{text} and can be specified in one of two ways.
Either by an initial ``\lstinline!:!'' followed by a sequence of \synt{text-character}s that
continues to the end of the line or by a ``\lstinline!%%!'' terminated by a ``\lstinline!%%!''
at the beginning of a line by itself. 
Text included using the ``\lstinline!:!'' notation will have trailing and leading 
whitespace removed.

ASDL has the following keywords:
\begin{quote}
  \lit{alias}  \lit{attributes} \lit{import}
  \lit{include} \lit{module} \lit{primitive} \lit{view}
\end{quote}%
Note that it is allowed to use a keyword as an identifier wherever a \synt{lc-id} is permitted.

\section{File Syntax}

An \asdl{} file consists of one or more \synt{definition}s possibly preceded by \lit{include}
directives.
A definition specifies either a module (see \secref{sec:module-syntax}),
primitive module (see \secref{sec:primitive-syntax}),
or view (see \secref{sec:view-syntax}).

Include directives allow one to split a large \asdl{} specification into multiple files,
while allowing \asdlgen{} to check references from one module to another.
\asdlgen{} will parse included files, but will not generate code for the definitions in included
files.
Also, included files will only be parsed once.

\begin{figure}[t]
  \begin{quote}
    \begin{grammar}
      <file>  ::=  \{ `include' <text> \} <definition> \{ <definition> \}
      
      <definition> ::= <module>
        \alt{} <primitive-module>
        \alt{} <view>
    \end{grammar}
  \end{quote}
  \caption{\asdl{} file syntax}
  \label{fig:file-syntax}
\end{figure}%

\section{Module Syntax}
\label{sec:module-syntax}

\figref{fig:module-syntax} gives the syntax for modules.
An \asdl{} module declaration consists of the keyword \lit{module}
followed by an identifier, an optional set of imported modules, and a
sequence of type definitions enclosed in braces.

\begin{figure}[t]
  \begin{quote}
    \begin{grammar}
      <module>  ::=  `module' <id> [ <imports> ] `{' \{ <type-definition> \} `}'

      <imports> ::=  `(' \{ `import' <id> [ `alias' <id> ] \} `)'
    \end{grammar}
  \end{quote}
  \caption{\asdl{} module syntax}
  \label{fig:module-syntax}
\end{figure}%

For example the
following example declares modules \lstinline[language=ASDL]!A!,
\lstinline[language=ASDL]!B!, and \lstinline[language=ASDL]!C!.
\lstinline[language=ASDL]!B! imports types from \lstinline[language=ASDL]!A!.
\lstinline[language=ASDL]!C! imports types from both \lstinline[language=ASDL]!A! and
\lstinline[language=ASDL]!B!.
Imports cannot be recursive; for example, it is an error for \lstinline[language=ASDL]!B! to
import \lstinline[language=ASDL]!C!, since \lstinline[language=ASDL]!C!
imports \lstinline[language=ASDL]!B!.
\begin{code}\begin{lstlisting}[language=ASDL]
 module A { ... } 
 module B (import A) { ... }
 module C (import A 
           import B) { ... }
\end{lstlisting}\end{code}% 

To refer to a type imported from another module the type must
\emph{always} be qualified by the module name from which it is
imported.
The following declares two different types called ``\texttt{t}.''
One in module \texttt{A} and one in module \texttt{B}.
The type ``\texttt{t}'' in module \texttt{B} defines a type ``\texttt{t}'' that
recursively mentions itself and also references the type ``\texttt{t}'' imported
from module \texttt{A}.
\begin{code}\begin{lstlisting}[language=ASDL]
module A { t = ... } 
module B (import A) { t = T(A.t, t) | N  ... }
\end{lstlisting}\end{code}% 

\section{Type Definitions}

The syntax of type definitions is given in \figref{fig:type-syntax}.
A type defintion begins with a type identifier, which is the name of the type.
The name must be unique within the module, but the order of definitions is unimportant.
When translating type definitions from a module they are placed in what would be considered a
module, package, or name-space of the same name.
If the output language does not support such features and only has one global
name space the module name
is used to prefix all the globally exported identifiers.

\begin{figure}[t]
  \begin{quote}
    \begin{grammar}
      <type-definition>  ::=  <typ-id> `=' <type>

      <type>         ::= <alias-type> | <sum-type> | <product-type>

      <alias-type>   ::= <typ-exp>
      
      <product-type> ::= <fields>

      <sum-type>     ::= <constructor> \{ `|' <constructor> \} [ `attributes' <fields> ]

      <constructor>  ::= <con-id> [ <fields> ]

      <fields>       ::= `(' \{ <field>  `,' \} <field> `)'

      <field>        ::=  <typ-exp> [ <id> ]
      
      <typ-exp>      ::= [ <id> `.' ] <typ-id> [ `?' | `*' | `!' ]
    \end{grammar}
  \end{quote}
  \caption{\asdl{} type definition syntax}
  \label{fig:type-syntax}
\end{figure}%

Type definitions are either alias types, which bind a name to a type expression;
product types, which are simple record definitions;
or sum type, which represent a discriminated union of possible values.
Unlike sum types, product types cannot form recursive type definitions, but they can
contain recursively declared sum types.

\subsection{Alias Types}

Alias types are the simplest form of type definition.
They provide a way to give a name to a type or type expression, similar
to \sml{}'s \lstinline[language=SML]@type@ and \Cplusplus{}'s
\lstinline[language=C++]@typedef@ constructs.

\subsection{Type expressions}

A type expression (\synt{typ-exp}) consists of a possibly qualified type name followed
by an optional type operator.
If the specified type is an ASDL primitive type or is defined in the current module,
then its name is not qualified; all other types defined outside the current module must
be qualified by their module name (or module alias).

The type operators are:
\begin{itemize}
  \item
    option (\lit{?}), which specifies either zero or one value of the specified type.
  \item
    sequence (\lit{*}), which specifies a sequence of zero or more values of the
    specified type, or
  \item
    shared (\lit{!}), which specifies that a value is shared across multiple points
    in the data structure (\ie{}, the structure has a DAG shape instead of just a tree).
\end{itemize}%

Note that while at most one type operator is allowed in a type expression, one can use alias
types to combine two or more operators.
For example, a sequence of optional integers could be defined by:
\begin{code}\begin{lstlisting}[language=ASDL]
int_opt = integer?
int_opt_seq = int_opt*
\end{lstlisting}\end{code}%

\subsection{ASDL Primitive Types}
There are seven pre-defined primitive types in ASDL, which are available without qualification:
\begin{description}
  \item[\normalfont\texttt{\color{\cdColor}bool}] describes Boolean values.
  \item[\normalfont\texttt{\color{\cdColor}int}] describes signed-integer values that are representable in 30 bits
    (\ie{}, in the range ${-}2^{29}$ to~\mbox{$2^{29}-1$}).
  \item[\normalfont\texttt{\color{\cdColor}uint}] describes unsigned-integer values that representable in 30 bits
    (\ie{}, in the range $0$ to~$2^{30}-1$).
  \item[\normalfont\texttt{\color{\cdColor}integer}] describes arbitrary-precision signed-integer values.
  \item[\normalfont\texttt{\color{\cdColor}natural}] describes arbitrary-precision unsigned-integer values.
  \item[\normalfont\texttt{\color{\cdColor}string}] describes length encoded strings of 8-bit characters.
  \item[\normalfont\texttt{\color{\cdColor}identifier}] describes strings with fast equality testing
    analogous to Lisp symbols.
\end{description}%

% TODO: Not implemented yet
%In addition, the \lstinline!StdTypes! module defines additional fixed-precision numeric types
%that can be used when necessary (\eg{}, \lstinline!StdTypes.int32!).

\subsection{Product Types}
Product types are record or tuple declarations.
They consist of a non-empty 
sequence of fields separated by commas enclosed in parenthesis. For
example 
\begin{code}\begin{lstlisting}[language=ASDL] 
pair_of_ints = (int, int) 
\end{lstlisting}\end{code}% 
declares a new type \lstinline[language=ASDL]!pair_of_ints! that consists of two integers.

\subsection{Sum Types}

Sum types are the most useful types in ASDL. They provide concise notation
used to describe a type that is the tagged union of a finite set of other
types.  Sum types consists of a series of constructors separated by a
vertical bar. Each constructor consist of a constructor identifier followed
by an optional sequence of fields similar to a product type. 

Constructor names must be unique within the module in which they are
declared. Constructors can be viewed as functions who take some number of
arguments of arbitrary type and create a value belonging to the sum type in
which they are declared.
For example
\begin{code}\begin{lstlisting}[language=ASDL]
module M {
  sexpr = Int(int)
	| String(string)
	| Symbol(identifier)
	| Cons(sexpr, sexpr)
	| Nil
}
\end{lstlisting}\end{code}%
declares that values of type \lstinline[language=ASDL]!sexpr! can either be
constructed from an \lstinline[language=ASDL]!int! using the
\lstinline[language=ASDL]!Int! constructor or a \lstinline[language=ASDL]!string!
from a \lstinline[language=ASDL]!String! constructor, an
\lstinline[language=ASDL]!identifier! using the \lstinline[language=ASDL]!Symbol!
constructor, or from two other \lstinline[language=ASDL]!sexpr! using the
\lstinline[language=ASDL]!Cons! constructor, or from no arguments
using the \lstinline[language=ASDL]!Nil! constructor.
Notice that the \lstinline[language=ASDL]!Cons! constructor
recursively refers to the \lstinline[language=ASDL]!sexpr! type.
\asdl{} allows sum types to be mutually recursive.
Recursion, however, is limited to sum types defined within the same module.

\subsubsection{Sum Types as Enumerations}
\label{sec:enumerations}

Sum types that consist completely of nullary constructors
are often treated specially and translated into static constants of a
enumerated value in languages that support them.
For example, the following \asdl{} specification:
%
\begin{code}\begin{lstlisting}[language=ASDL]
module Op {
  op = PLUS | MINUS | TIMES | DIVIDE 
}
\end{lstlisting}\end{code}%
%
Is translated into the following \Cplusplus{} code:
%
\begin{code}\begin{lstlisting}[language=c++]
namespace M {
    enum class op {
        PLUS = 1, MINUS, TIMES, DIVIDE
    };
}
\end{lstlisting}\end{code}%

\subsection{Fields}
A field consists of a type expression followed by an optional label.
The fields of a product or sum type must either all be labeled or unlabeled.
We use \emph{record} to refer to products of labeled fields and \emph{tuple}
to products of unlabeled fields.
Labels aid in the readability of
descriptions and are used by \asdlgen{} to name the fields of records
and classes for languages.
For example the declarations
\begin{code}\begin{lstlisting}[language=ASDL]
module Tree {
  tree = Node(int, tree, tree)
       | EmptyTree
}
\end{lstlisting}\end{code}%
can also be written as
\begin{code}\begin{lstlisting}[language=ASDL]
module Tree {
  tree = Node(int value, tree left, tree right)
       | EmptyTree
}
\end{lstlisting}\end{code}%

When translating the first definition without labels 
into \Cplusplus{} one would normally get
\begin{code}\begin{lstlisting}[language=c++,mathescape=true]
namespace Tree {
    $\cdots$
    class Node : public tree {
      public:
        Node (int p_value, tree * p_left, tree * p_right)
            : tree(tree::_con_Node), _v_value(p_value), _v_left(p_left),
            _v_right(p_right)
        { }
        ~Node ();
        void encode (asdl::outstream & os);
        int get_value () { return this->_v_value; }
        void set_value (int v) { this->_v_value = v; }
        tree * get_left () { return this->_v_left; }
        void set_left (tree * v) { this->_v_left = v; }
        tree * get_right () { return this->_v_right; }
        void set_right (tree * v) { this->_v_right = v; }
      private:
        int _v_value;
        tree * _v_left;
        tree * _v_right;
    };
    $\cdots$
} // namespace Tree
\end{lstlisting}\end{code}%
with labels one would get
\begin{code}\begin{lstlisting}[language=c++]
namespace Tree {
    $\cdots$
    struct Node : public tree {
        int value;
        tree *left;
        tree *right;
        Node (int v1, tree *v2, tree * v3)
          : tree(tree::_Node), value(v1), left(v2), right(v3) { }
        ~Node () { }
    };
    $\cdots$
}
\end{lstlisting}\end{code}%

For the \sml{} target, product types without labels are translated to tuples, while those with
labels are translated to records.

\subsection{Attribute Fields}
A sum-type definition may optionally be followed by a list of attribute fields, which
provide a concise way to specify fields that are common to all of the constructors
of a sum type.
For example, the definition
%
\begin{code}\begin{lstlisting}[language=ASDL]
module M {
  pos = (string file, int linenum, int charpos)
  sexpr = Int(int)
	| String(string)
	| Symbol(identifier)
	| Cons(sexpr, sexpr)
	| Nil
        attribute(pos)
}
\end{lstlisting}\end{code}%
adds a field of type \lstinline[language=ASDL]!pos! to all the constructors
in \lstinline[language=ASDL]!sexpr!.
One can think of an \lstinline!attribute! annotation as syntactic sugar for just including
the extra fields at the \emph{beginning} of each constructor's fields.
For example, the above definition can be viewed as syntactic sugar for
%
\begin{code}\begin{lstlisting}[language=ASDL]
module M {
  pos = (string file, int linenum, int charpos)
  sexpr = Int(pos, int)
	| String(pos, string)
	| Symbol(pos, identifier)
	| Cons(pos, sexpr, sexpr)
	| Nil(pos)
}
\end{lstlisting}\end{code}%
Note that this interpretation implies that attribute fields are labeled if, and only if, all
of the constructor fields are labeled.

Attribute fields are treated specially when translating to some targets.
For example in \Cplusplus{} code, the attribute field is defined in the base class for the sum type.

\section{Primitive Modules}
\label{sec:primitive-syntax}

\begin{figure}[t]
  \begin{quote}
    \begin{grammar}
      <primitive-module> ::= `primitive' `module' <id> `{' \{ <id> \} `}'
    \end{grammar}%
  \end{quote}%
  \caption{\asdl{} primitive module syntax}
  \label{fig:prim-module-syntax}
\end{figure}%

Primitive modules (see \figref{fig:prim-module-syntax}) provide a way to introduce abstract
types that are defined outside
of \asdl{} and which have their own pickling and unpickling code.
For example, we might want to include GUIDs (16-byte globally-unique IDs) in our pickles.
We can do so by first defining a primitive module \lstinline!Prim!:
%
\begin{quote}\begin{lstlisting}[language=ASDL]
primitive module Prim { guid }
\end{lstlisting}\end{quote}%
%
Then, depending on the target language, we define
supporting code to read and write guids from the byte stream.
In \sml{}, we would define three modules:
\begin{enumerate}
  \item \lstinline[language=SML]!structure Prim! that defines the representation of the
    \lstinline[language=SML]!guid! type.
  \item
    \lstinline[language=SML]!structure PrimPickle! that defines functions for pickling/unpickling
    a GUID to/from memory.
  \item
    \lstinline[language=SML]!structure PrimPickleIO! that defines functions for reading
    and writing GUIDs on binary I/O streams.
\end{enumerate}%
The \sml{} implementation of these modules could be written as follows:
\begin{code}\begin{lstlisting}[language=SML]
structure Prim : sig
    type guid
  end = struct
    type guid = GUID.guid
  end

structure PrimPickle : sig
    val guidEncode : Word8Buffer.buf * Prim.guid -> unit
    val guidDecode : Word8VectorSlice.slice
          -> Prim.guid * Word8VectorSlice.slice
  end = struct
    val guidSize = 16
    fun guidEncode (buf, guid) =
          Word8Buffer.addVec(buf, GUID.toBytes guid)
    fun guidDecode slice = let
    	  val (guid, rest) = Word8VectorSlice.splitAt(slice, guidSize)
	  in
	    (GUID.fromBytes(Word8VectorSlice.vector guid), rest)
	  end
   end

structure PrimPickleIO : sig
    val guidInput : BinIO.instream -> guid
    val guidOutput : BinIO.outstream * guid -> unit
  end = struct
    fun guidInput strm = (case GUID.fromBytes(BinIO.inputN(strm, 16))
	       of NONE => raise Fail "bogus GUID"
	        | (SOME g) => g
	      (* end case *))
    fun guidOutput (strm, g) = BinIO.output(strm, GUID.toBytes g)
  end
\end{lstlisting}\end{code}%
(assuming that the \lstinline!GUID! module implements the application's representation
of GUIDs).

For \Cplusplus{}, a primitive module requires a corresponding header file that declares
the primitive types and instances of the overloaded \lstinline[language=c++]!<<! and
\lstinline[language=c++]!>>! operators on the primitive types.  These declarations should
all be in a \lstinline[language=c++]!namespace! with the type name of the primitive module.
For example, the module from above would require the provision of a \texttt{Prim.hxx} header
file that contained something like the following code:
%
\begin{code}\begin{lstlisting}[language=c++]
#include <iostream>
#include "guid.hxx"

namespace Prim {

    typedef GUID::guid guid;

    std::istream &operator>> (std::istream &is, guid &g);
    std::ostream &operator<< (std::ostream &os, guid const &g);

}
\end{lstlisting}\end{code}%
(assuming that the \texttt{guid.hxx} header defines the application's representation
of GUIDs).
%For the \Cplusplus{} target, one should use string streams (in binary mode)
%to support pickling/unpickling of to/from memory.


\section{View Syntax}
\label{sec:view-syntax}

A view defines how an \asdl{} specification is translated to a target.
Each of the supported targets (\eg{}, \sml{} or \Cplusplus{}) has a default
view, but it is possible to customize the translation using \synt{view}
definitions.
The syntax of view declarations is given in \figref{fig:view-syntax}.
This section covers the syntax of views, but leaves the semantics to
\chapref{chap:views}.

\begin{figure}[t]
  \begin{quote}
    \begin{grammar}
      <view>        ::= `view' <id> `{' \{ <view-entry> \} `}'

      <view-entry>  ::=  <view-entities> `<=' <view-properties>
         \alt{} `<=' <id> `{' \{ <view-entity> <text> \} `}'

      <view-entities> ::= <view-entity>
         \alt{} `{' \{ <view-entity> \} `}'

      <view-entity> ::= `<file>'
         \alt{} `module' <id>
         \alt{} <id> `.' <typ-id> [ `.' `*' ]
         \alt{} <id> `.' <typ-id> `.' <con-id>

      <view-properties> ::= <id> <text>
          \alt{} `{' \{ <id> <text> \} `}'
    \end{grammar}%
  \end{quote}%
  \caption{\asdl{} view syntax}
  \label{fig:view-syntax}
\end{figure}%

\subsection{Basic View Syntax}
Views are named and consist of series of entries.
In its basic form, a view entry consists of a \synt{view-entity}, which specifies a file, module,
type, or constructor entity, and a view property, which is a name-value pair that is associated with
the entity.
%
\begin{quote}\begin{grammar}
  <view-entry>  ::= <view-entity> `<=' <id> <text>
\end{grammar}\end{quote}%
%
The meaning of an entry is to associate the specified view property with the specified view entity.
%%
%\begin{code}\begin{lstlisting}[language=ASDL]
%view Doc {
%  module  M <= doc_string
%%%
%  Types for representing LISP s-expressions.
%%%
%  M.sexpr  <= doc_string : s-expressions 
%  M.Int    <= doc_string : s-expression constructor
%  M.Symbol <= doc_string : s-expression constructor
%  M.Cons   <= doc_string : s-expression constructor
%  M.Nil    <= doc_string : s-expression constructor
%}
%
%view Java {
% M.sexpr <= source_name : Sexpr
% M.sexpr <= base_class  : MyClass
%}
%\end{lstlisting}\end{code}%
%%
%associates with the module \lstinline[language=ASDL]!M! the type
%\lstinline[language=ASDL]!M.sexpr! and the
%constructor \lstinline[language=ASDL]!M.Int!
%strings that will be added to the automatically generated documentation
%produced by the \texttt{--doc} command of \asdlgen{}.
%(\emph{In future we will probably dump them in comments in the output code too.})
%The view named \lstinline[language=ASDL]!Java! causes the type
%\lstinline[language=ASDL]!M.sexpr! to be renamed \lstinline[language=ASDL]!Sexpr! when
%generating Java output, and causes the abstract class normally generated to
%inherit from \lstinline[language=ASDL]!MyClass!. 

There can be multiple views with the same name.
The entries of two views with the same name are merged and consist of the
union of the entries in both.
It is an error, for two views of the same name to assign different values
to the same property of an entity.

\subsection{View Entry Derived Forms}
To make it easier to specify view entries, \asdl{} generalizes the basic syntax
to remove some of the redundancy of the basic syntax.
First, it is possible to specify multiple view entities on the left-hand-side of
the \lit{<=} symbol.
Likewise, it is possible to specify multiple view properties on the right-hand-side
of the \lit{<=} symbol.

It is also possible to assign different values to different entities for a fixed property
using the syntax.
%
\begin{quote}\begin{grammar}
  <view-entry> ::= `<=' <id> `{' \{ <view-entity> <text> \} `}'
\end{grammar}\end{quote}%
%
Here the property name is given first, followed by a sequence of view-entity-value pairs.

Lastly, \asdl{} allows a \lit{.*} suffix to be added to sum-type entities.
This suffix means that the entity specifies the set of all of the constructors of the type.

%Examples of the sugared notation are shown
%below in their respective order.
%\begin{code}\begin{lstlisting}[language=ASDL]
%view Doc {
% { M.Int  M.Symbol  
%   M.Cons M.Nil } <= doc_string : s-expression constructor
%
% <= doc_string {
%  module  M 
%%%
%  Types for representing LISP s-expressions.
%%%
%  M.sexpr : s-expressions 
%  }
%}
%
%view Cxx {
%  M.sexpr <= {
%    source_name : Sexpr
%    base_class  : MyClass
%  }
%}
%\end{lstlisting}\end{code}%
