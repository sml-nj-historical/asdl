%!TEX root = manual.tex
%
\chapter{Code Interface}
\label{sec:code-interface}

In this section, we describe the default translation of ASDL definitions to target
languages and describe some of the runtime assumptions that users need to be aware of
when using the generated code.

\section{Translation to \sml{}}

The translation from an \asdl{} specification to \sml{} code is straightforward.
\asdl{} modules map to \sml{} structures, \asdl{} product types map to either tuples or records,
and \asdl{} sum types map the \sml{} datatypes.
\tblref{tbl:asdl-to-sml} summarizes this translation.
If an \asdl{} identifier conflicts with an \sml{} keyword or pervasive identifier, then the translation
adds a trailing prime character (\lstinline[language=SML]!'!) to the identifier.

For an \asdl{} module \lstinline[mathescape=true]@$M$@, we generate three \sml{} structures:
\begin{code}\begin{lstlisting}[language=SML,mathescape=true]
structure $M$ = struct ... end

structure $M$Pickle = struct ... end

structure $M$PickleIO = struct ... end
\end{lstlisting}\end{code}
where \lstinline[language=SML,mathescape=true]@$M$@ structure contains the type definitions
for the \asdl{} specification, \lstinline[language=SML,mathescape=true]@$M$Pickle@ structure
implements functions to convert between the types and byte vectors, and the
\lstinline[language=SML,mathescape=true]@$M$PickleIO@ structure implements functions to
read and write pickles from binary files.

\begin{table}[tp]
  \caption{Translation of \asdl{} types to \sml{}}
  \label{tbl:asdl-to-sml}
  \begin{center}
    \begin{tabular}{|l|p{3in}|}
      \hline
      \textbf{\asdl{} type} & \textbf{\sml{} type} \\
      \hline
      \textit{Named types ($T$)} &  ($\hat{T}$) \\[0.25em]
      \lstinline!bool! & \lstinline[language=SML]!bool! \\[0.5em]
      \lstinline!int! & \lstinline[language=SML]!int! \\[0.5em]
      \lstinline!uint! & \lstinline[language=SML]!word! \\[0.5em]
      \lstinline!integer! & \lstinline[language=SML]!IntInf.int! \\[0.5em]
      \lstinline!string! & \lstinline[language=SML]!string! \\[0.5em]
      \lstinline!identifier! & \lstinline[language=SML]!Atom.atom! \\[0.5em]
      \lstinline[language=ASDL,mathescape=true]@$t$@ & \lstinline[language=SML,mathescape=true]!$t$! \\[0.5em]
      \lstinline[language=ASDL,mathescape=true]@$M$.$t$@ & \lstinline[language=SML,mathescape=true]!$M$.$t$! \\[0.5em]
      \hline
      \textit{Type expressions ($\tau$)} &  ($\hat{\tau}$) \\[0.25em]
      \lstinline[language=ASDL,mathescape=true]@$T$@ & \lstinline[language=SML,mathescape=true]!$\hat{T}$! \\[0.5em]
      \lstinline[language=ASDL,mathescape=true]@$T$?@ & \lstinline[language=SML,mathescape=true]!$\hat{T}$ option! \\[0.5em]
      \lstinline[language=ASDL,mathescape=true]@$T$*@ & \lstinline[language=SML,mathescape=true]!$\hat{T}$ list! \\[0.5em]
      \hline
      \textit{Product types ($\rho$)} & ($\hat{\rho}$) \\[0.25em]
      \lstinline[language=ASDL,mathescape=true]@($\tau_1$, $\ldots$, $\tau_n$)@
        & \lstinline[language=SML,mathescape=true]!$\hat{\tau}_1$ * $\cdots$ * $\hat{\tau}_n$! \\[0.5em]
      \lstinline[language=ASDL,mathescape=true]@($\tau_1$ $f_1$, $\ldots$, $\tau_n$ $f_n$)@
        & \lstinline[language=SML,mathescape=true]!{$f_1$ : $\hat{\tau}_1$, $\ldots$, $f_n$ : $\hat{\tau}_n$}! \\[0.5em]
      \hline
      \textit{Type definitions} & \\[0.25em]
      \lstinline[language=ASDL,mathescape=true]@$t$ = $\rho$@
        & \lstinline[language=SML,mathescape=true]!type $t$ = $\hat{\rho}$! \\[0.5em]
      \lstinline[language=ASDL,mathescape=true]@$t$ = $C_1$ | $\cdots$ | $C_n$@
        & \lstinline[language=SML,mathescape=true]!datatype $t$ = $C_1$ | $\cdots$ | $C_n$! \\[0.5em]
      \lstinline[language=ASDL,mathescape=true]@$t$ = $C_1$($\rho_1$) | $\cdots$ | $C_n$($\rho_n$)@
        & \lstinline[language=SML,mathescape=true]!datatype $t$ = $C_1$ of $\hat{\rho}_1$ | $\cdots$ | $C_n$ of $\hat{\rho}_n$! \\[0.25em]
      \hline
    \end{tabular}%
  \end{center}%
\end{table}%

\section{Translation to \Cplusplus{}}

The translation of an \asdl{} specification to \Cplusplus{} is more complicated than for \sml{}.
For each \asdl{} module, we define a corresponding \Cplusplus{} namespace.

\begin{table}[tp]
  \caption{Translation of \asdl{} types to \Cplusplus{}}
  \label{tbl:asdl-to-cxx}
  \begin{center}
    \begin{tabular}{|p{2in}|p{3in}|}
      \hline
      \textbf{ASDL type} & \textbf{\Cplusplus{} type} \\
      \hline
      \textit{Named types ($T$)} &  ($\hat{T}$) \\[0.25em]
      \lstinline!bool! & \lstinline[language=c++]!bool! \\[0.5em]
      \lstinline!int! & \lstinline[language=c++]!int! \\[0.5em]
      \lstinline!uint! & \lstinline[language=c++]!unsigned int! \\[0.5em]
      \lstinline!integer! & \lstinline[language=c++]!asdl::integer! \\[0.5em]
      \lstinline!string! & \lstinline[language=c++]!std::string! \\[0.5em]
      \lstinline!identifier! & \lstinline[language=c++]!asdl::identifier! \\[0.5em]
      \lstinline[language=ASDL,mathescape=true]@$t$@ &
        $\left\{
        \begin{array}{ll}
          t & \text{if $t$ is an \lstinline[language=c++]!enum! type} \\
          \text{\lstinline[language=c++,mathescape=true]@$t$ *@} & otherwise \\
        \end{array}
        \right.$ \\[0.5em]
      \lstinline[language=ASDL,mathescape=true]@$M$.$t$@ & \lstinline[language=c++,mathescape=true]@$M$::$t$@ \\[0.5em]
      \hline
      \textit{Type expressions ($\tau$)} &  ($\hat{\tau}$) \\[0.25em]
      \lstinline[language=ASDL,mathescape=true]@$T$@ & \lstinline[language=c++,mathescape=true]@$\hat{T}$@ \\[0.5em]
      \lstinline[language=ASDL,mathescape=true]@$T$?@ & \lstinline[language=c++,mathescape=true]@asdl::option< $\hat{T}$ >@ \\[0.5em]
      \lstinline[language=ASDL,mathescape=true]@$T$*@ & \lstinline[language=c++,mathescape=true]@std::vector< $\hat{T}$ >@ \\[0.5em]
      \hline
      \textit{Product types ($\rho$)} & ($\hat{\rho}$) \\[0.25em]
      \lstinline[language=ASDL,mathescape=true]@($\tau_1$, $\ldots$, $\tau_n$)@
        & \lstinline[language=c++,mathescape=true]@$\hat{\tau}_1$ _v1; $\ldots$ $\hat{\tau}_n$ _vn@ \\[0.5em]
      \lstinline[language=ASDL,mathescape=true]@($\tau_1$ $f_1$, $\ldots$, $\tau_n$ $f_n$)@
        & \lstinline[language=c++,mathescape=true]@$\hat{\tau}_1$ _$f_1$; $\ldots$ $\hat{\tau}_n$ _$f_n$@ \\[0.5em]
      \hline
      \textit{Type definitions} & \\[0.25em]
      \lstinline[language=ASDL,mathescape=true]@$t$ = $\rho$@
        & \lstinline[language=c++,mathescape=true]@struct $t$ { $\hat{\rho}$ };@ \\[0.5em]
      \lstinline[language=ASDL,mathescape=true]@$t$ = $C_1$ | $\cdots$ | $C_n$@
        & \lstinline[language=c++,mathescape=true]@class enum $t$ { $C_1$, $\ldots$, $C_n$ };@ \\[0.5em]
      \lstinline[language=ASDL,mathescape=true]@$t$ = $C_1$($\rho_1$) | $\cdots$ | $C_n$($\rho_n$)@
        &
\vspace*{-1em}
\begin{lstlisting}[language=c++,mathescape=true]
class $t$ { $\cdots$ };
class $C_1$ : public $t$ {
  private: $\hat{\rho}_1$
  $\cdots$
};
$\cdots$
class $C_n$ : public $t$ {
  private: $\hat{\rho}_n$
  $\cdots$
};
\end{lstlisting}%
      \\[0.25em]
      \hline
    \end{tabular}%
  \end{center}%
\end{table}%

\subsection{Memory management}


\section{The Rosetta Stone for Sum Types}
\label{sec:rosetta-stone}

In languages that have algebraic data types sum types are equivalent to the
\lstinline[language=ASDL]!datatype! and \lstinline[language=ASDL]!data! declarations in ML and Haskell. In Algol like
languages they are equivalent to tagged \lstinline[language=ASDL]!unions! in C or variant records
in Pascal. In class based object oriented languages sum types are equivalent
to an abstract base class that represents the type of a family of subclasses
one for each constructor of the type. The previous example written in ML would
be
\begin{quote}\begin{lstlisting}[language=SML]
structure M =
  struct
    datatype sexpr
      = Int of (int)
      | String of (string)
      | Symbol of (identifier)
      | Cons of (sexpr * sexpr)
      | Nil
  end
\end{lstlisting}\end{quote}%
and in \Cplusplus{} it translates to
\begin{quote}\begin{lstlisting}[language=c++]
namespace M {

    struct sexpr {
        enum tag {
            _Int, _String, _Symbol, _Cons, _Nil
        };
        tag _tag;
        sexpr (tag t) : _tag(t) { }
        virtual ~sexpr ();
    };

    struct Int : public sexpr {
        int _v1;
        Int (int v) : sexpr(sexpr::_Int), _v1(v) { }
        ~Int () { }
    };

    struct String : public sexpr {
        std::string _v1;
        String (const char *v) : sexpr(sexpr::_String), _v1(v) { }
        String (std::string const &v) : sexpr(sexpr::_String), _v1(v) { }
        ~String () { }
    };

    struct Symbol : public sexpr { ... };

    struct Cons : public sexpr { ... };

    struct Nil : public sexpr { ... };

}
\end{lstlisting}\end{quote}%
